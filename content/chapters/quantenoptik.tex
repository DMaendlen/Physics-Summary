\chapter{Quantenoptik}
	\section{Photonenenergie}
		Im Grenzfall gilt $W_{\mathrm{Austritt}} = E_{\mathrm{Grenz}}$
		\[ E_{ph} = h f = \frac{hc}{\lambda} = \frac{h'}{\lambda} \]
		\begin{table}[h]
		\begin{tabular}{ll}
		$h=6,626\cdot10^{-34}$Js & $h'\dots$ 1,24$\mu$meV\\
		\end{tabular}
		\end{table}
		
	\section{Impuls $p$ eines Photons}
		\[p = \frac{hf}{c} = \frac{h}{\lambda} = m v\]
		
		Photonenmasse:
		\begin{align*}
			m &= \frac{E_{ph}}{c^2} = \frac{hf}{c^2} &&& E_{kin} &= E - E_0 = (m - m_0)c^2 \\
			v &= c \sqrt{1-\frac{1}{1+(\frac{eU_A}{m_o c^2})^2}} &&&
		\end{align*}
		
		$c\dots$ Lichtgeschwindigkeit, üblicherweise $3\cdot10^8\nicefrac{\mathrm{m}}{\mathrm{s}}$
		
	\section{Strahlungsleistung $\Phi$ eines Lichtstrahles}
		\begin{align*}
			\Phi &= E_{\mathrm{ph}} \dot{N}_{\mathrm{ph}} = IA 
			& \dot{N} &= \frac{\lambda}{hc} \Phi = \frac{\lambda}{h'} \Phi 
			& I_{\mathrm{ph}} &= e\dot{N} \\
			\Phi & = \Phi_0 \mathrm{e}^{-\alpha d} 
		\end{align*}
		
		\begin{table}[h]
		\begin{tabular}{lll}
		$I_{ph}\dots$ Strom & $\dot{N}\dots$ Photonenstrom & $I\dots$ Intensität $\left[\frac{W}{m^2}\right]$ \\
		$A\dots$ Fläche & $\alpha\dots$ Absorbtionskoeffizient & $d\dots$ Dicke des Kristalls\\
		$h'\dots$ 1,24$\mu$meV\\
		\end{tabular}
		\end{table}
		
	\section{Beugung von Lichtwellen}
		oder Heisenbergsche Unschärferelation genannt
		\begin{align*}
			e U_A &= \frac12 m v^2 & v &= \sqrt{\frac{2eU_A}{m}}
		\end{align*}
		Beugung bei kleinen Winkeln $\tan \alpha = \frac{\Delta px}{py} = \frac{\Delta px}{\frac{h}{\lambda}} $\newline
		Beugungsminimum: $\sin \alpha = \frac{\lambda}{b} = \frac{\lambda}{\Delta x}$ \newline
		Bei kleinen Winkeln: $\tan x  = \sin x $ \newline
		\begin{align*}
			\frac{\Delta px}{\frac{h}{\lambda}} &= \frac{\lambda}{\Delta x} & \Delta px \Delta x &= h
		\end{align*}

		\subsection{Brechung, Reflexion}
			\begin{align*}
				\varrho_{\mathrm{s}}&=\left(\frac{n_2-n_1}{n_2+n_1}\right)^2 
				& \tau_{\mathrm{s}}&=\frac{4n_1n_2}{(n_1+n_2)^2}=\frac{I_{\mathrm{t}}}{I_{\mathrm{e}}}
				& \alpha_{\mathrm{s}}&=\frac{I_{\mathrm{a}}}{I_{\mathrm{e}}}
			\end{align*}

			\begin{table}[h]
			\begin{tabular}{lll}
			$\varrho_{\mathrm{s}}\dots$ Reflexionsgrad & $\tau_{\mathrm{s}}\dots$ Transmissionsgrad & $\alpha_{\mathrm{s}}\dots$ Absorptionsgrad\\
			\end{tabular}
			\end{table}
			\begin{table}[h]
			\begin{tabular}{ll}
			Stoff & Brechzahl\\
			\toprule
			Wasser & $n_{\mathrm{H_2O}}=1,33$\\
			Glas & $n_{\mathrm{Glas}}=1,5$\\
			Luft & $n_{\mathrm{Luft}}=1$\\
			\end{tabular}
			\caption{Häufige Brechzahlen}
			\end{table}

	\section{Elektronen im elektrischen Feld}
		\begin{align*}
			F_{\text{el}}&= -eE & a=-\frac{e}{m}E
		\end{align*}

		Arbeit: $W_{12}=eU_{12}=\frac{1}{2}m(v_2^2-v_1^2)$
		Wellenlänge des Elektrons:

		\begin{align*}
			E_{\text{kin}}&= eU_A & \frac{1}{2}mv^2&= eU_A & v=\sqrt{\frac{2eU_A}{m}}
		\end{align*}

		$h=6,626\cdot10{-34}$Js \\
		mit p=mv=$\nicefrac{h}{\lambda}$ folgt die Wellenlänge $\lambda=f(U_A)$
		\[
			\lambda=\frac{h}{\sqrt{2eU_Am}}
		\]
		\begin{table}[h]
		\begin{tabular}{ll}
		$U_A\dots$ Beschleunigungsspannung & $m\dots$ Elektronenmasse\\
		\end{tabular}
		\end{table}

	\section{Bahnkurve des Elektrons im E-Feld}
		\[
			a_x=0; a_y=\frac{e}{m}E; v_x=v_x;v_y=\frac{e}{m}Et
		\]
		\begin{align*}
			V_x&= v_xt & v_y&= \frac{1}{2}\frac{l}{m}Et^2 & y&= \frac{1}{2}\frac{e}{m}E\frac{x^2}{v^2} & y&= \frac{1}{4}\frac{U_K}{U_A}\frac{x^2}{d}
		\end{align*}
		\begin{table}[h]
		\begin{tabular}{ll}
		$d\dots$ Plattenabstand & $e=1,602\cdot10^{-19}$As\\
		\end{tabular}
		\end{table}
	
	\section{Bewegung von Elektronen im Magnetfeld}
		\[
			F_{\text{L}}=qvB \Rightarrow e^-:F_{\text{L}}=-evB
		\]
		\[
			F_{\mathrm{L}}=F_{\mathrm{Zp}} \Rightarrow qvB=\frac{mv^2}{r}
		\]
		\subsection{Linke-Hand-Regel}
			\begin{tabularx}{\textwidth}{l l}
				Daumen & Elektronenflussrichtung\\
				Zeigefinger & Magnetfeldrichtung\\
				Mittelfinger & Lorentzkraft\\
			\end{tabularx}
        
		\subsection{Rechte-Hand-Regel}
			\begin{tabularx}{\textwidth}{l l}
				Daumen & technische Stromrichtung\\
				Zeigefinger & Magnetfeldrichtung\\
				Mittelfinger & Lorentzkraft\\
			\end{tabularx}
	
	\section{Radius im Fadenstrahlrohr}
		\begin{align*}
			\frac{mv^2}{r}&= evB \Rightarrow r = \frac{mv}{eB} & T= \frac{2\pi r}{v} &= \frac{2\pi m}{eB}
		\end{align*}

		\begin{align*}
			T &= \frac{2\pi}{B}\frac{m}{e} & \omega &= \frac{2\pi}{T}=\frac{eB}{m}
		\end{align*}

		\subsection{Zyklotron}
			\begin{align*}
				\tau &= \frac{\mu m^*}{e} 
				& v_\text{{th}} &= \sqrt{\frac{3kT}{m^*}} 
				& l &= v_{th}\tau 
				& f_{\text{res}} &= \frac{1}{\tau} 
				& B_{\text{res}} &= \frac{2\pi f_{\text{res}} m^*}{e}
			\end{align*}

			\begin{table}[h]
			\begin{tabular}{ll}
			$\tau\dots$ Mittlere Flugdauer(Streuzeit), Relaxationszeit & $m^*\dots$ Reduzierte Masse\\
			$l\dots$ mittlere freie Weglänge & $v_{\text{th}}\dots$ thermische Geschwindigkeit \\
			$f_{\text{res}}\dots$ Zyklotronenresonanzfrequenz & $B\dots$ Magnetfeld, [T]\\
			\end{tabular}
			\end{table}

	\section{Leiterschleife im Magnetfeld}
		\begin{align*}
			M&=IBA\sin\alpha=IBab\sin\alpha & ab&=A
		\end{align*}

	\section{Relativistische Bewegung}
		\begin{align*}
			E &= mc^2\text{(ruhend)} & \text{bzw. } E_0 &= m_0c^2 & m &= \frac{m_0}{\sqrt{1-\frac{v^2}{c^2}}}
		\end{align*}

	\section{Elektronen im Atom}
		Kreisfrequenz eines Elektrons $\Rightarrow F_{\text{Z}}=F_{\text{coulomb}}$ 
		wobei $Q=q_1q_2=Ze^2$, allerdings nur bei Atomen, die nur ein Außenelektron haben.
		\begin{align*}
			\omega^2 &= \frac{Ze^2}{4\pi\varepsilon_0m_0r^3} 
			& E_{\text{kin}}=\frac{1}{2}J\omega^2=\frac{1}{2}mr^2\omega^2=\frac{Ze^2}{8\pi\varepsilon_0}\frac{l}{r}
		\end{align*}
		\begin{align*}
			E_{\text{pot}} &= -\frac{1}{4\pi\varepsilon_0}\frac{Ze^2}{r} 
			& E_{\text{ges}} &= E_{\text{kin}}+E_{\text{pot}} = -\frac{Ze^2}{8\pi\varepsilon_0}\frac{l}{r}
		\end{align*}
	
		\begin{table}[h]
		\begin{tabular}{ll}
		$Z\dots$ Ordnungszahl\\
		\end{tabular}
		\end{table}
	\section{Bohrsches Postulat}
		für stehende Wellen gilt:
		\[
			\text{Bahnumfang} = 2\pi r = n\lambda = n\lambda = n\frac{h}{p} \Rightarrow l = rp = n\frac{h}{2\pi}
		\]
		\[
			l = n\frac{h}{2\pi} = n\bar{h} = m_0r\omega
		\]

		mit $\omega^2=\dots$ ergibt sich:
		\[
			r_n=\frac{4\pi\varepsilon_0\bar{h}^2}{e^2m_0}\frac{n^2}{Z}
		\]
		mit $e^2m_0=5,29\cdot10^{-11}$m.

		Die Energie eines Elektrons auf einer Bahn
		\[
			E_n=-Z^2\frac{e^4m_0}{32\pi^2\varepsilon_0^2\bar{h}^2}\frac{1}{n^2}
		\]
		mit $32\pi^2\varepsilon_0^2\bar{h}^2=13,6$eV.

		\subsection{Bindungs- oder Ionisierungsenergie}
		$\frac{e^4m_0}{32\pi^2\varepsilon_0^2\bar{h}^2}$ ist ein konstanter Faktor = 13,6 eV, gegebenenfalls 
		mit reduzierter Masse multiplizieren oder durch $\varepsilon^2$ dividieren.

	\section{Fermi-Energie}
		\[
			E_{\text{F}}=\frac{\bar{h}^2}{2m}(3\pi^2)^{\frac{2}{3}}n^{\frac{2}{3}} \text{ mit } n=\frac{N}{V}
		\]
        
		Übergang zur Fermi-Geschwindigkeit:
		\[
			E_{\text{F}}=\frac{\bar{h}k_f}{2m}=\frac{p_{\text{F}}^2}{2m}=\frac{1}{2}mv_{\text{F}}^2
		\]

		\begin{table}[h]
		\begin{tabular}{lll}
		$n\dots$ Ladungsträgerdichte & $N\dots$ Anzahl der Ladungsträger & $\bar{h}=1,055\cdot10^{-34}\mathrm{Js}=6,58\cdot10^{-16}\mathrm{eVs}$\\
		\end{tabular}
		\end{table}
        
		\[
			v_{\text{F}}=\sqrt{\frac{2E_{\text{F}}}{m}}
		\]
       
       		\begin{table}[h]
		\begin{tabular}{ll} 
		$v_{\text{F}}\dots$ Fermigeschwindigkeit & $k_{\text{F}}\dots$ Wellenzahl, $k=\frac{2\pi}{\lambda}$\\
		\end{tabular}
		\end{table}
        
		\[
			v_{\text{F}}=\frac{\bar{h}}{m}(3\pi^2n)^{\frac{1}{3}}
		\]

	\section{Zustandsdichte bei $T=0$K}
		\[
			D(E)=\frac{1}{2\pi^2}\left(\frac{2m^{\frac{2}{3}}}{\bar{h}}\right)E^{\frac{1}{2}}
		\]

		Mit zunehmender Temperatur gilt die Besetzungswahrscheinlichkeit
		\[
			f(E)=\frac{1}{1+e^{\frac{E-E_{\text{F}}}{kT}}}
		\]
	
	\section{Drehmoment im homogenen elektrischen Feld}
		\begin{align*}
			M &= Fr\sin\beta & \vec{M}&=\vec{p}\times\vec{E} & \vec{p}&= q\vec{r} \text{ el. Dipolmoment}
		\end{align*}

	\section{Drehmoment im magnetischen Feld}
		\begin{align*}
			\vec{m}&=I\vec{A} \text{ mag. Dipolmoment} & m&=N\mu_B \text{ mag. Dipolmoment}
		\end{align*}

		\begin{table}[h]
		\begin{tabular}{ll}
		$\mu_B=9,274\cdot10^{-24}\text{Am}^2\dots$ Bohrsches Magneton & $N\dots$ Ladungsträger\\
		\end{tabular}
		\end{table}
