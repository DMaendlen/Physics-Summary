\chapter{Einheiten}
	\begin{table}[ht]
	\begin{tabularx}{\textwidth}{lll}
	Größe & Abkürzung & Einheiten\\
	\toprule
	Elektrische Feldkonstante & $\varepsilon_0$ 
	& $\frac{\mathrm{As}}{\mathrm{m}}=\frac{\mathrm{C}}{\mathrm{Vm}}=\frac{\mathrm{C}^2}{\mathrm{Nm}^2}$\\
	\midrule
	Elektrische Feldstärke & $E$ 
	& $\frac{\mathrm{V}}{\mathrm{m}}=\frac{\mathrm{VN}}{\mathrm{Ws}}=\frac{\mathrm{N}}{\mathrm{As}}=\frac{\mathrm{N}}{\mathrm{C}}=\frac{\mathrm{kgm}^2}{\mathrm{s}^3\mathrm{A}}$\\
	\midrule
	Elektrische Kapazität & $C$ 
	& $F=\frac{\mathrm{C}}{\mathrm{V}}=\frac{\mathrm{As}}{\mathrm{V}}=\frac{\mathrm{s}}{\Omega}=\frac{\mathrm{As}^4}{\mathrm{kgm}^2}=\frac{\mathrm{Nm}}{\mathrm{V}^2}=\frac{\mathrm{kgm}^2}{\mathrm{V}^2\mathrm{s}^2}$\\
	\midrule
	Elektrische Ladung & $Q$ & C = AS = $\frac{\mathrm{Ws}}{\mathrm{V}}=\frac{\mathrm{Nm}}{\mathrm{V}}=\frac{\mathrm{kgm}^2}{\mathrm{Vs}^2}$\\
	\midrule
	Elektrische Spannung & $U$ & V = A$\Omega = \frac{\mathrm{W}}{\mathrm{A}} = \frac{\mathrm{Nm}}{\mathrm{As}} = \frac{\mathrm{Ws}}{\mathrm{As}} = \frac{\mathrm{J}}{\mathrm{As}} = \frac{\mathrm{kgm}^2}{\mathrm{as}^3}$\\
	\midrule
	Elektrische Stromstärke & $I$ & A = $\frac{\mathrm{V}}{\Omega}$\\
	\midrule
	Elektrischer Widerstand & $R$ & $\Omega=\frac{\mathrm{V}}{\mathrm{A}}=\frac{\mathrm{kgm}^2}{\mathrm{A}^2\mathrm{s}^3}$\\
	\midrule
	Energie & $E$/$W$ & J = VAs = Ws = Nm = $\frac{\mathrm{kgm}^2}{\mathrm{s}^2}=\frac{1}{1,602\cdot10^{-19}}$eV\\
	\midrule
	Kraft & $F$ & N = $\frac{\mathrm{VAs}}{\mathrm{m}}=\frac{\mathrm{Ws}}{\mathrm{m}}=\frac{\mathrm{J}}{\mathrm{m}}=\frac{\mathrm{kgm}}{\mathrm{s}^2}$\\
	\midrule
	Leistung & $P$ & W = VA = $\frac{\mathrm{J}}{\mathrm{s}}=\frac{\mathrm{Nm}}{\mathrm{s}}=\frac{\mathrm{kgm}^2}{\mathrm{s}^3}$\\
	\midrule
	Magnetische Flussdichte & $B$ & T = $\frac{\mathrm{kg}}{\mathrm{As}^2}=\frac{\mathrm{Vs}}{\mathrm{m}^2}$\\
	\bottomrule
	\end{tabularx}
	\label{Einheiten}
	\caption{Einheitentabelle, Quelle: HMS, 12. Auflage}
	\end{table}
